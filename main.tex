\documentclass{article}
\usepackage[utf8]{inputenc}
\usepackage[spanish]{babel}
\usepackage{listings}
\usepackage{graphicx}
\graphicspath{ {images/} }
\usepackage{cite}

\begin{document}

\begin{titlepage}
    \begin{center}
        \vspace*{1cm}
            
        \Huge
        \textbf{Proyecto Final - Los primeros pasos}
            
        \vspace{0.5cm}
        \LARGE
        
            
        \vspace{1.5cm}
            
        \textbf{Sebastían Boívar Vanegas 1017268527}
            
        \vfill
            
        \vspace{0.8cm}
            
        \Large
        Despartamento de Ingeniería Electrónica y Telecomunicaciones\\
        Universidad de Antioquia\\
        Medellín\\
        Marzo de 2021
            
    \end{center}
\end{titlepage}

\tableofcontents
\newpage
\section{Ideas proyecto final.}\label{intro}
De inicio me gustaría conformar un grupo de 2 integrantes pero de momento me encuentro solo, ya que no he tenido mucha interacción con los demás compañeros del curso; pero de entrada quiero contextualizar mis ideas y si logro formar un equipo, unir las ideas del compañero y asi afianzar todo en un común de ideas que sean de agrado para los dos.\\


\textbf{Ideas principales:} \\
\\
\textbf{*} Quisiera realizar un video juego en el cual el ambiente musical y visual sea de agrado para cualquier persona que le de inicio al juego. \\
\\
\textbf{*} Implementar lo aprendido en el desarrollo del curso para que el juego sea de calidad. \\
\\
\textbf{*} Me gustaría que el juego pertenezca a una categoría de aventuras.\\
\\
\textbf{*} Intentar que el juego sea multijugador de manera local, pero si no lo logro realizar por lo menos que funcione de buena manera para un único jugador\\
\\
\textbf{*} Quiero que el jugador principal o jugadores principales tengan una barra de salud que se vaya disminuyendo con el contacto de enemigos o al recibir impactos de dichos enemigos. \\
\\
\textbf{*} Los jugadores que puedan realizar saltos para sobrepasar obstáculos que el mapa tenga.\\ 
\\
\textbf{*} Quiero que los players tengan como objetivo recolectar ciertos objetos. Estos objetos manejaran un tema de aleatoriedad en la cual no siempre te va a salir un objeto beneficioso que te de un plus de velocidad o saltabilidad, también que salgan reductores de vida o incluso que reduzcan la movilidad para así agregarle una pizca de emoción y dificultad al juego.\\
\\
\textbf{*} Al aniquilar enemigos quiero que se de un puntaje y al conseguir cierto puntaje que aparezca un boss que sea de mayor dificultad vencerle para así poder conseguir una llave que me de ingreso al siguiente nivel. \\
\\
\textbf{*} Los participantes contaran con un total de 2 vidas, las cuales no se pueda incrementar, únicamente reducir. En si el enfoque para permanecer con vida en el juego es supervisar la barra de salud para así no perder vidas, lo cual dependerá del transcurso del juego.\\
\\
\textbf{*} Quiero que los jugadores o jugador tengan una habilidad de poder atacar a los enemigos y al boss. También me gustaría que el daño que genera este ataque y el estilo de ataque también se tenga en cuenta en los objetos que se van encontrando en el mapa. Ya que vaya variando su daño de impacto o su estilo tanto como beneficio como perjuicio y que así se siga cumpliendo un nivel de aleatoriedad y dificultad constante. \\
\\
\textbf{*} Me gustaría agregar animaciones, pero no sé si por el tiempo en el semestre pueda crearlas e integrarle al video juego, pero en un futuro seria interesante agregarlas y así darle un plus al ambiente y a la historia que se vaya desarrollando en el juego.\\
\\

\textbf{*} Quiero implementarle bases físicas al programa para que las acciones tomadas tengan un acercamiento mas a la realidad y no se vea como algo tan poco posible.\\










\end{document}
